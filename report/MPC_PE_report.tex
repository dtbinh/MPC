% --------------------------------------------------------------
% This is all preamble stuff that you don't have to worry about.
% Head down to where it says "Start here"
% --------------------------------------------------------------
 
\documentclass[11pt]{article}
 
\usepackage[margin=1in]{geometry} 
\usepackage{amsmath,amsthm,amssymb,amsfonts}
\usepackage{tikz}
   \usetikzlibrary{calc,positioning}
   \tikzset{>=latex}
\usepackage{pgfplots}
\usepackage{bm}
\usepackage{xcolor}
\usepackage{units}
\usepackage{floatrow}
\usepackage{times}
\usepackage{epsfig}
\usepackage{graphicx}
\usepackage{subcaption}
\usepackage[nomarkers]{endfloat}
% \usepackage[numbered, framed]{mcode}
 
\newcommand{\N}{\mathbb{N}}
\newcommand{\Z}{\mathbb{Z}}
\newcommand{\R}{\mathbb{R}}
\newcommand{\X}{\mathcal{X}}
\newcommand{\Xf}{\mathcal{X}_f}
 
\newenvironment{theorem}[2][Theorem]{\begin{trivlist}
\item[\hskip \labelsep {\bfseries #1}\hskip \labelsep {\bfseries #2.}]}{\end{trivlist}}
\newenvironment{lemma}[2][Lemma]{\begin{trivlist}
\item[\hskip \labelsep {\bfseries #1}\hskip \labelsep {\bfseries #2.}]}{\end{trivlist}}
\newenvironment{exercise}[2][Exercise]{\begin{trivlist}
\item[\hskip \labelsep {\bfseries #1}\hskip \labelsep {\bfseries #2.}]}{\end{trivlist}}
\newenvironment{reflection}[2][Reflection]{\begin{trivlist}
\item[\hskip \labelsep {\bfseries #1}\hskip \labelsep {\bfseries #2.}]}{\end{trivlist}}
\newenvironment{proposition}[2][Proposition]{\begin{trivlist}
\item[\hskip \labelsep {\bfseries #1}\hskip \labelsep {\bfseries #2.}]}{\end{trivlist}}
\newenvironment{corollary}[2][Corollary]{\begin{trivlist}
\item[\hskip \labelsep {\bfseries #1}\hskip \labelsep {\bfseries #2.}]}{\end{trivlist}}
 
\begin{document}
 
% --------------------------------------------------------------
%                         Start here
% --------------------------------------------------------------
 
%\renewcommand{\qedsymbol}{\filledbox}
 
\title{Model Predictive Control\\
    Programming Exercise - Report} %replace X with the appropriate number
\author{Gian Andrea M{\"u}ller (14-935-035)\\
    Kevin Anschau Schwarzer (12-914-735)\\
    Ueli Eugen Wechsler (11-920-444)} %replace with your name
%if necessary, replace with your course title

\maketitle

\begin{enumerate}
    % 1.
    \item Interpretation of the matrices $A_c$ and $B_c$:
    
    \begin{enumerate}
    \item
    Rows \textbf{1-3} and \textbf{7-9} of $A_c$ describe how the position / the angle change depending on their respective velocities.
        
    \item Rows \textbf{4} and \textbf{5} of $A_c$ describe the change of the velocity in x- and y-direction depending on gravitational acceleration.
    
    \begin{align}
    \ddot{x}&=9.81\cdot\beta \label{ddotx} \\
    \ddot{y}&=-9.81\cdot\alpha \label{ddoty} \\
    \ddot{z}&=u_1+u_2+u_3+u_4 \label{ddotz}
    \end{align}
    
	Non-zero pitch- and roll-angles result in a non-zero projection of the gravitational acceleration on the body-fixed x- and y-axis. This projection is represented by equations (\ref{ddotx},\ref{ddoty}).
	
Since the system is linearized in $(x_s,u_s)$ such that the effect of gravity is compensated for entirely, $\dot{z}$ is unaffected by gravity even with non-zero roll- and pitch angles.
	\item 
	The sixth row of $B_c$ refers to acceleration of the craft in z-direction depending on the combined thrust of all 4 rotors. (\ref{ddotz})
	\item Row \textbf{10-11} describe how the imbalance between opposing rotors results in an angular acceleration in $\alpha$ and $\beta$. (\ref{ddotalpha},\ref{ddotbeta})
	
	\begin{alignat}{6}
	\ddot{\alpha}=& & 0.56\cdot u_2 &\null &-0.56\cdot u_4 \label{ddotalpha} \\
	\ddot{\beta} = & -0.56\cdot u_1 &&+0.56\cdot u_3 &\label{ddotbeta} \\
	\ddot{\gamma}= & 0.73\cdot u_1&-0.73\cdot u_2&+0.73\cdot u_3&-0.73\cdot u_4 \label{ddotgamma}
	\end{alignat}
	
	\item The last row accounts for the fact that the two rotor pairs turn in different directions and therefore induce a rotation about the z-axis as soon as their respective torques do not add up to zero. (\ref{ddotgamma})
    \end{enumerate}
% subsection nonlinear_model_and_linearization (end)
\end{enumerate}

\newpage
\subsection*{First MPC controller} % (fold)
\label{sub:first_mpc_controller}

\begin{enumerate}
    \setcounter{enumi}{1}
    % 2.
    \item Choice of tuning parameters $(Q,R,P,A_{\Xf},b_{\Xf})$
    
    \begin{align}
    Q &= \text{diag}(10000,\ 4000,\ 9000,\ 5,\ 250,\ 200,\ 1)\\
    R &= 0.1*\text{eye(4)}
    \end{align}

	Reasoning for the chosen Q and R matrices:    
    \begin{itemize}
    \item \textbf{Cheap Control}: Select R to be significantly smaller than Q in order to allow aggressive control action.
    \item \textbf{Prioritize Q}: Chose the respective weights of $\dot{z}$, $\alpha$ and $\beta$ in Q dramatically larger in order to reduce their errors fast. The asymmetry in $\alpha$ and $\beta$ compensates for the different convergence of the two and results in a relatively symmetric reaction for all following applications.
    
    \vspace{2ex}
    
    In addition, to help with the reduction of errors in $\alpha$ and $\beta$ the weights of $\dot{\alpha}$ and $\dot{\beta}$ are increased slightly.
    
    \vspace{2ex}
    
    Finally the weight of $\gamma$ is chosen relatively low since only a slow regulation is required there.
    \end{itemize}
    
    \begin{equation}
    P=\begin{bmatrix}1&0&0&0&0&0&0\\0&1.419&0&0&0.056&0&0&\\0&0&2.291&0&0&0.077&0\\0&0&0&0.0034&0&0&0\\0&0.056&0&0&0.0296&0&0\\0&0&0.077&0&0&0.0258&0\\0&0&0&0&0&0&0.0004\end{bmatrix}
    \end{equation}
    
    The P-matrix shown above results from the implementation of an LQR-controller based on the system model as well as Q and R.
    
    \begin{verbatim}
    [K_inf,P_inf,~] = dlqr(sys.A,sys.B,Q,R);
    \end{verbatim}
    
    Consequently we apply the resulting control-law to get a closed loop system. After imposing state- and input constraints we are ultimately able to calculate the final set which in this case is the \textbf{LQR maximum control invariant set}.

    % 3.
    \item The response of the first MPC controller can be seen in
    Figure~\ref{fig:1st_mpc_controller}.
    \begin{figure}[ht]
        \centering
        \begin{subfigure}[c]{0.3\linewidth}
            \centering
            \includegraphics[width=\linewidth]{Plots_03_FirstMPCController/01}
            \caption{Roll- and Pitch}
        \end{subfigure}
        ~
        \begin{subfigure}[c]{0.3\linewidth}
            \centering
            \includegraphics[width=\linewidth]{Plots_03_FirstMPCController/02}
            \caption{Yaw}
        \end{subfigure}
        ~
        \begin{subfigure}[c]{0.3\linewidth}
            \centering
            \includegraphics[width=\linewidth]{Plots_03_FirstMPCController/03}
            \caption{Rotor Speeds}
        \end{subfigure}

        \begin{subfigure}[c]{0.3\linewidth}
            \centering
            \includegraphics[width=\linewidth]{Plots_03_FirstMPCController/04}
            \caption{zdot}
        \end{subfigure}
        ~
        \begin{subfigure}[c]{0.3\linewidth}
            \centering
            \includegraphics[width=\linewidth]{Plots_03_FirstMPCController/05}
            \caption{Roll and Pitch rates}
        \end{subfigure}
        ~
        \begin{subfigure}[c]{0.3\linewidth}
            \centering
            \includegraphics[width=\linewidth]{Plots_03_FirstMPCController/06}
            \caption{Yaw rate}
        \end{subfigure}
        \caption{Response of first MPC controller.}
        \label{fig:1st_mpc_controller}
\end{figure}
\end{enumerate}

% subsection first_mpc_controller (end)


\newpage
\subsection*{Reference Tracking} % (fold)
\label{sub:reference_tracking}

\begin{enumerate}
    \setcounter{enumi}{3}
    % 4.
    \item Define the state $(x_r,u_r)$ as a function of an arbitrary reference.
    
    To find the target state $(x_r,u_r)$ we remind ourselves that we want to find some desirable state $x_r$ and a corresponding input signal $u_r$ which result in a steady state of the system and produce an output $y=r$. Formulated mathematically this results in the following system of equations.
    
    \begin{align}
    \underbrace{\begin{bmatrix}I-A&-B\\C&0\end{bmatrix}}_{\text{LHS}}\begin{bmatrix}x_r\\u_r\end{bmatrix}&=\begin{bmatrix}0\\r\end{bmatrix}\\
    \end{align}
    
    To solve the system above the matrix on the left hand side is inverted and multiplied with the right hand side. In matlab this works as follows:
    
    \begin{verbatim}
    TargetState = LHS\[zeros(nx,1); ref];
    xr = TargetState(1:nx);
    ur = TargetState(nx+1:end);
    \end{verbatim}
    
    Herein the sdpvariable \verb|ref| is used to make the calculation depend on any reference signal handed over later by simQuad.

    % 5.
    \item The response for the constant reference signal can be seen in
    Figure~\ref{fig:constant_reference_with_offset}.
    \begin{figure}[ht]
        \centering
        \begin{subfigure}[c]{0.3\linewidth}
            \centering
            \includegraphics[width=\linewidth]{Plots_05_ReferenceTracking_Constant/01}
            \caption{Roll and Pitch}
        \end{subfigure}
        ~
        \begin{subfigure}[c]{0.3\linewidth}
            \centering
            \includegraphics[width=\linewidth]{Plots_05_ReferenceTracking_Constant/02}
            \caption{Yaw}
        \end{subfigure}
        ~
        \begin{subfigure}[c]{0.3\linewidth}
            \centering
            \includegraphics[width=\linewidth]{Plots_05_ReferenceTracking_Constant/03}
            \caption{Rotor Speeds}
        \end{subfigure}

        \begin{subfigure}[c]{0.3\linewidth}
            \centering
            \includegraphics[width=\linewidth]{Plots_05_ReferenceTracking_Constant/04}
            \caption{zdot}
        \end{subfigure}
        ~
        \begin{subfigure}[c]{0.3\linewidth}
            \centering
            \includegraphics[width=\linewidth]{Plots_05_ReferenceTracking_Constant/05}
            \caption{Roll and Pitch rates}
        \end{subfigure}
        ~
        \begin{subfigure}[c]{0.3\linewidth}
            \centering
            \includegraphics[width=\linewidth]{Plots_05_ReferenceTracking_Constant/06}
            \caption{Yaw rate}
        \end{subfigure}
        \caption{Response for constant reference signal with offset.}
        \label{fig:constant_reference_with_offset}
\end{figure}

    % 6.
    \item The response for the slowly varying reference signal can be seen in
    Figure~\ref{fig:varying_reference_with_offset}.
    \begin{figure}[ht]
        \centering
        \begin{subfigure}[c]{0.3\linewidth}
            \centering
            \includegraphics[width=\linewidth]{Plots_06_ReferenceTracking_Varying/01}
            \caption{Roll and Pitch}
        \end{subfigure}
        ~
        \begin{subfigure}[c]{0.3\linewidth}
            \centering
            \includegraphics[width=\linewidth]{Plots_06_ReferenceTracking_Varying/02}
            \caption{Yaw}
        \end{subfigure}
        ~
        \begin{subfigure}[c]{0.3\linewidth}
            \centering
            \includegraphics[width=\linewidth]{Plots_06_ReferenceTracking_Varying/03}
            \caption{Rotor Speeds}
        \end{subfigure}

        \begin{subfigure}[c]{0.3\linewidth}
            \centering
            \includegraphics[width=\linewidth]{Plots_06_ReferenceTracking_Varying/04}
            \caption{zdot}
        \end{subfigure}
        ~
        \begin{subfigure}[c]{0.3\linewidth}
            \centering
            \includegraphics[width=\linewidth]{Plots_06_ReferenceTracking_Varying/05}
            \caption{Roll and Pitch rates}
        \end{subfigure}
        ~
        \begin{subfigure}[c]{0.3\linewidth}
            \centering
            \includegraphics[width=\linewidth]{Plots_06_ReferenceTracking_Varying/06}
            \caption{Yaw rate}
        \end{subfigure}
        \caption{Response for a slowly varying reference signal with offset.}
        \label{fig:varying_reference_with_offset}
\end{figure}
\end{enumerate}

% subsection reference_tracking (end)


\subsection*{First simulation of the nonlinear model} % (fold)
\label{sub:first_simulation_of_the_nonlinear_model}

\begin{enumerate}
    \setcounter{enumi}{6}
    % 7.
    \item The response of the reference tracking of the nonlinear model can be
    seen in Figure~\ref{fig:nonlinear_reference_tracking_with_offset}.
    \begin{figure}[ht]
        \centering
        \begin{subfigure}[c]{0.3\linewidth}
            \centering
            \includegraphics[width=\linewidth]{Plots_07_NonlinearModel_ReferenceTracking/01}
            \caption{Quadrotor}
        \end{subfigure}
        ~
        \begin{subfigure}[c]{0.3\linewidth}
            \centering
            \includegraphics[width=\linewidth]{Plots_07_NonlinearModel_ReferenceTracking/02}
            \caption{x, y and z}
        \end{subfigure}
        ~
        \begin{subfigure}[c]{0.3\linewidth}
            \centering
            \includegraphics[width=\linewidth]{Plots_07_NonlinearModel_ReferenceTracking/03}
            \caption{Roll and Pitch}
        \end{subfigure}

        \begin{subfigure}[c]{0.3\linewidth}
            \centering
            \includegraphics[width=\linewidth]{Plots_07_NonlinearModel_ReferenceTracking/04}
            \caption{zdot}
        \end{subfigure}
        ~
        \begin{subfigure}[c]{0.3\linewidth}
            \centering
            \includegraphics[width=\linewidth]{Plots_07_NonlinearModel_ReferenceTracking/05}
            \caption{Yaw}
        \end{subfigure}
        ~
        \begin{subfigure}[c]{0.3\linewidth}
            \centering
            \includegraphics[width=\linewidth]{Plots_07_NonlinearModel_ReferenceTracking/06}
            \caption{Rotor Speeds}
        \end{subfigure}

        \begin{subfigure}[c]{0.3\linewidth}
            \centering
            \includegraphics[width=\linewidth]{Plots_07_NonlinearModel_ReferenceTracking/07}
            \caption{xdot and ydot}
        \end{subfigure}
        ~
        \begin{subfigure}[c]{0.3\linewidth}
            \centering
            \includegraphics[width=\linewidth]{Plots_07_NonlinearModel_ReferenceTracking/08}
            \caption{Roll and Pitch rate}
        \end{subfigure}
        ~
        \begin{subfigure}[c]{0.3\linewidth}
            \centering
            \includegraphics[width=\linewidth]{Plots_07_NonlinearModel_ReferenceTracking/09}
            \caption{Yaw rate}
        \end{subfigure}
        \caption{Reference tracking response of the nonlinear model.}
        \label{fig:nonlinear_reference_tracking_with_offset}
\end{figure}
\end{enumerate}

% subsection first_simulation_of_the_nonlinear_model (end)


\subsection*{Offset free MPC} % (fold)
\label{sub:offset_free_mpc}

\begin{enumerate}
    \setcounter{enumi}{7}
    % 8.
    \item The matrix L:
    
    Our calculation of L is based on the solution of the dual LQR-problem for the augmented system.
    
    \begin{verbatim}
    L = dlqr(A_aug',C_aug',Q_aug,R_aug)';
    \end{verbatim}
    
    For this purpose special weighting matrices $Q_{aug}$ and $R_{aug}$ were designed. Those were chosen as follows:
    
    \begin{verbatim}
    Q_aug = diag([ones(1,nx) [50 1 1 500 10 10 0.01]);
    R_aug = eye(nx);
    \end{verbatim}
    
    Reasoning for the selection of $Q_{aug}$ and $R_{aug}$:
    
    \begin{itemize}
    \item 
    \end{itemize}
    
    \begin{equation}
    L = \begin{bmatrix} 
    0.8966&0&0&0&0&0&0\\
    0&0.5808&0&0&0.0317&0&0\\
    0&0&0&0.9821&0&0&0.0307\\
    0&0.0015&0&0&0.7758&0&0\\
    0&0&0.0015&0&0&0.7758&0\\
    0&0&0&0.0001&0&0&0.3316\\
    0.7191&0&0&0&0&0&0\\
    0&0.2050&0&0&0.0017&0&0\\
    0&0&0.2050&0&0&0.0017&0\\
    0&0&0&0.9458&0&0&-0.0211\\
    0&-0.0106&0&0&0.4734&0&0\\
    0&0&-0.0106&0&0&0.4734&0\\
    0&0&0&0&0&0&0.0259\\
    \end{bmatrix}
    \end{equation}
    
	% 9.
	\item The response of the offset free tracking with a constant reference can be
    seen in Figure~\ref{fig:offset_free_tracking_with_constant}.
    \begin{figure}[ht]
        \centering
        \begin{subfigure}[c]{0.3\linewidth}
            \centering
            \includegraphics[width=\linewidth]{Plots_09_OffsetFreeTracking_Constant/01}
            \caption{Roll and Pitch}
        \end{subfigure}
        ~
        \begin{subfigure}[c]{0.3\linewidth}
            \centering
            \includegraphics[width=\linewidth]{Plots_09_OffsetFreeTracking_Constant/02}
            \caption{Yaw}
        \end{subfigure}
        ~
        \begin{subfigure}[c]{0.3\linewidth}
            \centering
            \includegraphics[width=\linewidth]{Plots_09_OffsetFreeTracking_Constant/03}
            \caption{Rotor Speeds}
        \end{subfigure}

        \begin{subfigure}[c]{0.3\linewidth}
            \centering
            \includegraphics[width=\linewidth]{Plots_09_OffsetFreeTracking_Constant/04}
            \caption{zdot}
        \end{subfigure}
        ~
        \begin{subfigure}[c]{0.3\linewidth}
            \centering
            \includegraphics[width=\linewidth]{Plots_09_OffsetFreeTracking_Constant/05}
            \caption{Roll and Pitch rates}
        \end{subfigure}
        ~
        \begin{subfigure}[c]{0.3\linewidth}
            \centering
            \includegraphics[width=\linewidth]{Plots_09_OffsetFreeTracking_Constant/06}
            \caption{Yaw rate}
        \end{subfigure}

        \begin{subfigure}[c]{0.3\linewidth}
            \centering
            \includegraphics[width=\linewidth]{Plots_09_OffsetFreeTracking_Constant/08}
            \caption{zdot disturbance}
        \end{subfigure}
        ~
        \begin{subfigure}[c]{0.3\linewidth}
            \centering
            \includegraphics[width=\linewidth]{Plots_09_OffsetFreeTracking_Constant/09}
            \caption{Yawdot disturbance}
        \end{subfigure}
        ~
        \begin{subfigure}[c]{0.3\linewidth}
            \centering
            \includegraphics[width=\linewidth]{Plots_09_OffsetFreeTracking_Constant/10}
            \caption{Disturbance of $\dot{\alpha}$ and $\dot{\beta}$}
        \end{subfigure}
        \caption{Offset-free tracking with a constant reference.}
        \label{fig:offset_free_tracking_with_constant}
\end{figure}

    % 10.
    \item The response of the offset free tracking with a varying reference can be
    seen in Figure~\ref{fig:offset_free_tracking_with_varying}.
    \begin{figure}[ht]
        \centering
        \begin{subfigure}[c]{0.3\linewidth}
            \centering
            \includegraphics[width=\linewidth]{Plots_10_OffsetFreeTracking_Varying/01}
            \caption{Roll and Pitch}
        \end{subfigure}
        ~
        \begin{subfigure}[c]{0.3\linewidth}
            \centering
            \includegraphics[width=\linewidth]{Plots_10_OffsetFreeTracking_Varying/02}
            \caption{Yaw}
        \end{subfigure}
        ~
        \begin{subfigure}[c]{0.3\linewidth}
            \centering
            \includegraphics[width=\linewidth]{Plots_10_OffsetFreeTracking_Varying/03}
            \caption{Rotor Speeds}
        \end{subfigure}

        \begin{subfigure}[c]{0.3\linewidth}
            \centering
            \includegraphics[width=\linewidth]{Plots_10_OffsetFreeTracking_Varying/04}
            \caption{zdot}
        \end{subfigure}
        ~
        \begin{subfigure}[c]{0.3\linewidth}
            \centering
            \includegraphics[width=\linewidth]{Plots_10_OffsetFreeTracking_Varying/05}
            \caption{Roll and Pitch rates}
        \end{subfigure}
        ~
        \begin{subfigure}[c]{0.3\linewidth}
            \centering
            \includegraphics[width=\linewidth]{Plots_10_OffsetFreeTracking_Varying/06}
            \caption{Yaw rate}
        \end{subfigure}

        \begin{subfigure}[c]{0.3\linewidth}
            \centering
            \includegraphics[width=\linewidth]{Plots_10_OffsetFreeTracking_Varying/08}
            \caption{zdot disturbance}
        \end{subfigure}
        ~
        \begin{subfigure}[c]{0.3\linewidth}
            \centering
            \includegraphics[width=\linewidth]{Plots_10_OffsetFreeTracking_Varying/09}
            \caption{Yawdot disturbance}
        \end{subfigure}
        ~
        \begin{subfigure}[c]{0.3\linewidth}
            \centering
            \includegraphics[width=\linewidth]{Plots_10_OffsetFreeTracking_Varying/10}
            \caption{Disturbance of $\dot{\alpha}$ and $\dot{\beta}$}
        \end{subfigure}
        \caption{Offset-free tracking with a varying reference.}
        \label{fig:offset_free_tracking_with_varying}
\end{figure}
    
\end{enumerate}

% subsection offset_free_mpc (end)


\subsection*{Simulation of the nonlinear model} % (fold)
\label{sub:simulation_of_the_nonlinear_model}

\begin{enumerate}
    \setcounter{enumi}{10}
    % 11.
    \item The response of the nonlinear model to a step signal can be
    seen in Figure~\ref{fig:nonlinear_model_step_signal}.
    \begin{figure}[ht]
        \centering
        \begin{subfigure}[c]{0.3\linewidth}
            \centering
            \includegraphics[width=\linewidth]{Plots_11_NonlinearModel_StepSignal/01}
            \caption{Quadrotor}
        \end{subfigure}
        ~
        \begin{subfigure}[c]{0.3\linewidth}
            \centering
            \includegraphics[width=\linewidth]{Plots_11_NonlinearModel_StepSignal/02}
            \caption{x, y and z}
        \end{subfigure}
        ~
        \begin{subfigure}[c]{0.3\linewidth}
            \centering
            \includegraphics[width=\linewidth]{Plots_11_NonlinearModel_StepSignal/03}
            \caption{Roll and Pitch}
        \end{subfigure}
        
        \begin{subfigure}[c]{0.3\linewidth}
            \centering
            \includegraphics[width=\linewidth]{Plots_11_NonlinearModel_StepSignal/04}
            \caption{zdot}
        \end{subfigure}
        ~
        \begin{subfigure}[c]{0.3\linewidth}
            \centering
            \includegraphics[width=\linewidth]{Plots_11_NonlinearModel_StepSignal/05}
            \caption{Yaw}
        \end{subfigure}
        ~
        \begin{subfigure}[c]{0.3\linewidth}
            \centering
            \includegraphics[width=\linewidth]{Plots_11_NonlinearModel_StepSignal/06}
            \caption{Rotor Speeds}
        \end{subfigure}

        \begin{subfigure}[c]{0.3\linewidth}
            \centering
            \includegraphics[width=\linewidth]{Plots_11_NonlinearModel_StepSignal/07}
            \caption{xdot and ydot}
        \end{subfigure}
        ~
        \begin{subfigure}[c]{0.3\linewidth}
            \centering
            \includegraphics[width=\linewidth]{Plots_11_NonlinearModel_StepSignal/08}
            \caption{Roll and Pitch rates}
        \end{subfigure}
        ~
        \begin{subfigure}[c]{0.3\linewidth}
            \centering
            \includegraphics[width=\linewidth]{Plots_11_NonlinearModel_StepSignal/09}
            \caption{Yaw Rate}
        \end{subfigure}
        
        \begin{subfigure}[c]{0.3\linewidth}
            \centering
            \includegraphics[width=\linewidth]{Plots_11_NonlinearModel_StepSignal/10}
            \caption{zdot disturbance}
        \end{subfigure}
        ~
        \begin{subfigure}[c]{0.3\linewidth}
            \centering
            \includegraphics[width=\linewidth]{Plots_11_NonlinearModel_StepSignal/11}
            \caption{Disturbance of $\dot{\alpha}$ and $\dot{\beta}$}
        \end{subfigure}
        
        \caption{Response of the nonlinear model with step signal.}
        \label{fig:nonlinear_model_step_signal}
\end{figure}

    % 12.
    \item The response of the nonlinear model to a hexagon signal can be
    seen in Figure~\ref{fig:nonlinear_model_hexagon_signal}.
    \begin{figure}[ht]
        \centering
        \begin{subfigure}[c]{0.3\linewidth}
            \centering
            \includegraphics[width=\linewidth]{Plots_12_NonlinearModel_Hexagon/01}
            \caption{Quadrotor}
        \end{subfigure}
        ~
        \begin{subfigure}[c]{0.3\linewidth}
            \centering
            \includegraphics[width=\linewidth]{Plots_12_NonlinearModel_Hexagon/02}
            \caption{x, y, and z}
        \end{subfigure}
        ~
        \begin{subfigure}[c]{0.3\linewidth}
            \centering
            \includegraphics[width=\linewidth]{Plots_12_NonlinearModel_Hexagon/03}
            \caption{Roll and Pitch}
        \end{subfigure}
        
        \begin{subfigure}[c]{0.3\linewidth}
            \centering
            \includegraphics[width=\linewidth]{Plots_12_NonlinearModel_Hexagon/04}
            \caption{zdot}
        \end{subfigure}
        ~
        \begin{subfigure}[c]{0.3\linewidth}
            \centering
            \includegraphics[width=\linewidth]{Plots_12_NonlinearModel_Hexagon/05}
            \caption{Yaw}
        \end{subfigure}
        ~
        \begin{subfigure}[c]{0.3\linewidth}
            \centering
            \includegraphics[width=\linewidth]{Plots_12_NonlinearModel_Hexagon/06}
            \caption{Rotor Speeds}
        \end{subfigure}

        \begin{subfigure}[c]{0.3\linewidth}
            \centering
            \includegraphics[width=\linewidth]{Plots_12_NonlinearModel_Hexagon/07}
            \caption{xdot and ydot}
        \end{subfigure}
        ~
        \begin{subfigure}[c]{0.3\linewidth}
            \centering
            \includegraphics[width=\linewidth]{Plots_12_NonlinearModel_Hexagon/08}
            \caption{Roll and Pitch rates}
        \end{subfigure}
        ~
        \begin{subfigure}[c]{0.3\linewidth}
            \centering
            \includegraphics[width=\linewidth]{Plots_12_NonlinearModel_Hexagon/09}
            \caption{Yaw rate}
        \end{subfigure}
        
        \begin{subfigure}[c]{0.3\linewidth}
            \centering
            \includegraphics[width=\linewidth]{Plots_12_NonlinearModel_Hexagon/10}
            \caption{zdot disturbance}
        \end{subfigure}
        ~
        \begin{subfigure}[c]{0.3\linewidth}
            \centering
            \includegraphics[width=\linewidth]{Plots_12_NonlinearModel_Hexagon/11}
            \caption{Disturbance of $\dot{\alpha}$ and $\dot{\beta}$}
        \end{subfigure}
        
        \caption{Nonlinear model with hexagon signal.}
        \label{fig:nonlinear_model_hexagon_signal}
\end{figure}

    % 13.
    \item The response of the nonlinear model to a lemniscate signal can be
    seen in Figure~\ref{fig:nonlinear_model_lemniscate_signal}.
    \begin{figure}[ht]
        \centering
        \begin{subfigure}[c]{0.3\linewidth}
            \centering
            \includegraphics[width=\linewidth]{Plots_13_NonlinearModel_Lemniscate/01}
            \caption{Quadrotor}
        \end{subfigure}
        ~
        \begin{subfigure}[c]{0.3\linewidth}
            \centering
            \includegraphics[width=\linewidth]{Plots_13_NonlinearModel_Lemniscate/02}
            \caption{x, y, and z}
        \end{subfigure}
        ~
        \begin{subfigure}[c]{0.3\linewidth}
            \centering
            \includegraphics[width=\linewidth]{Plots_13_NonlinearModel_Lemniscate/03}
            \caption{Roll and Pitch}
        \end{subfigure}
        
        \begin{subfigure}[c]{0.3\linewidth}
            \centering
            \includegraphics[width=\linewidth]{Plots_13_NonlinearModel_Lemniscate/04}
            \caption{zdot}
        \end{subfigure}
        ~
        \begin{subfigure}[c]{0.3\linewidth}
            \centering
            \includegraphics[width=\linewidth]{Plots_13_NonlinearModel_Lemniscate/05}
            \caption{Yaw}
        \end{subfigure}
        ~
        \begin{subfigure}[c]{0.3\linewidth}
            \centering
            \includegraphics[width=\linewidth]{Plots_13_NonlinearModel_Lemniscate/06}
            \caption{Rotor Speeds}
        \end{subfigure}

        \begin{subfigure}[c]{0.3\linewidth}
            \centering
            \includegraphics[width=\linewidth]{Plots_13_NonlinearModel_Lemniscate/07}
            \caption{xdot and ydot}
        \end{subfigure}
        ~
        \begin{subfigure}[c]{0.3\linewidth}
            \centering
            \includegraphics[width=\linewidth]{Plots_13_NonlinearModel_Lemniscate/08}
            \caption{Roll and Pitch rates}
        \end{subfigure}
        ~
        \begin{subfigure}[c]{0.3\linewidth}
            \centering
            \includegraphics[width=\linewidth]{Plots_13_NonlinearModel_Lemniscate/09}
            \caption{Yaw rate}
        \end{subfigure}
        
        \begin{subfigure}[c]{0.3\linewidth}
            \centering
            \includegraphics[width=\linewidth]{Plots_13_NonlinearModel_Lemniscate/10}
            \caption{zdot disturbance}
        \end{subfigure}
        ~
        \begin{subfigure}[c]{0.3\linewidth}
            \centering
            \includegraphics[width=\linewidth]{Plots_13_NonlinearModel_Lemniscate/11}
            \caption{Disturbance of $\dot{\alpha}$ and $\dot{\beta}$}
        \end{subfigure}
        
        \caption{Nonlinear model with lemniscate signal.}
        \label{fig:nonlinear_model_lemniscate_signal}
\end{figure}
\end{enumerate}

% subsection simulation_of_the_nonlinear_model (end)


\subsection*{Slew rate constraints} % (fold)
\label{sub:slew_rate_constraints}

\begin{enumerate}
    \setcounter{enumi}{13}
    % 14.
    \item The response of the offset-free tracking with slew rate constraints can be
    seen in Figure~\ref{fig:slew_rate}.
    \begin{figure}[ht]
        \centering
        \begin{subfigure}[c]{0.3\linewidth}
            \centering
            \includegraphics[width=\linewidth]{Plots_14_SlewRateConstraints/01}
            \caption{Roll and Pitch}
        \end{subfigure}
        ~
        \begin{subfigure}[c]{0.3\linewidth}
            \centering
            \includegraphics[width=\linewidth]{Plots_14_SlewRateConstraints/02}
            \caption{Yaw}
        \end{subfigure}
        ~
        \begin{subfigure}[c]{0.3\linewidth}
            \centering
            \includegraphics[width=\linewidth]{Plots_14_SlewRateConstraints/03}
            \caption{Rotor Speeds}
        \end{subfigure}

        \begin{subfigure}[c]{0.3\linewidth}
            \centering
            \includegraphics[width=\linewidth]{Plots_14_SlewRateConstraints/04}
            \caption{zdot}
        \end{subfigure}
        ~
        \begin{subfigure}[c]{0.3\linewidth}
            \centering
            \includegraphics[width=\linewidth]{Plots_14_SlewRateConstraints/05}
            \caption{Roll and Pitch rates}
        \end{subfigure}
        ~
        \begin{subfigure}[c]{0.3\linewidth}
            \centering
            \includegraphics[width=\linewidth]{Plots_14_SlewRateConstraints/06}
            \caption{Yaw rate}
        \end{subfigure}

        \begin{subfigure}[c]{0.3\linewidth}
            \centering
            \includegraphics[width=\linewidth]{Plots_14_SlewRateConstraints/08}
            \caption{zdot disturbance}
        \end{subfigure}
        ~
        \begin{subfigure}[c]{0.3\linewidth}
            \centering
            \includegraphics[width=\linewidth]{Plots_14_SlewRateConstraints/09}
            \caption{Yawdot disturbance}
        \end{subfigure}
        ~
        \begin{subfigure}[c]{0.3\linewidth}
            \centering
            \includegraphics[width=\linewidth]{Plots_14_SlewRateConstraints/10}
            \caption{Disturbance of $\dot{\alpha}$ and $\dot{\beta}$}
        \end{subfigure}
        \caption{Offset-free tracking with slew rate constraints.}
        \label{fig:slew_rate}
\end{figure}
\end{enumerate}

% subsection slew_rate_constraints (end)


\subsection*{Soft Constraints} % (fold)
\label{sub:soft_constraints}

\begin{enumerate}
    \setcounter{enumi}{14}
    % 15.
    \item

    % 16.
    \item

\end{enumerate}

% subsection soft_constraints (end)


\subsection*{FORCES Pro} % (fold)
\label{sub:forces_pro}

\begin{enumerate}
    \setcounter{enumi}{16}
    % 17.
    \item
\end{enumerate}

% subsection forces_pro (end)

\end{document}
